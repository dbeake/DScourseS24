\documentclass{article}

% Language setting
% Replace `english' with e.g. `spanish' to change the document language
\usepackage[english]{babel}

% Set page size and margins
% Replace `letterpaper' with `a4paper' for UK/EU standard size
\usepackage[letterpaper,top=2cm,bottom=2cm,left=3cm,right=3cm,marginparwidth=1.75cm]{geometry}

% Useful packages
\usepackage{amsmath}
\usepackage{graphicx}
\usepackage[colorlinks=true, allcolors=blue]{hyperref}

\title{Problem Set 6}
\author{Dillon Beake}

\begin{document}
\maketitle


\section{Overview}

Three different visualizations were created for this project. I used data from the 2007 NFL Draft. The Visualizations are included below.  
Before I move on, I want to discuss any aspects of the data I had to clean.  
\linespread{1.6} % Double spacing
To clean the data I took 3 approaches.  
\linespread{1.6} % Double spacing
1) For the Scatter Plot, I excluded any entries that had an "NA", instead of a value that I wanted to include.  this was written in the code, as a command line. 
\linespread{1.6} % Double spacing
2) For the Box Plot I deleted any item that had only 1 example.  For example, there were a few positions that were coded wrong, i.e. "S = safety", instead of "SS", or "FS", strong safety and free safety respectively. This was written in the code as a  command line. This was done to make the visualization easier to read and it did not take away from the message I wanted it to convey.  
\linespread{1.6} % Double spacing
3) I noticed in my Data that there were Draft Rounds that did not exist.  To correct the draft rounds I went through the data and manually changed the draft round for each player. For future research.  I will find data NFL Draft data that at least has the correct draft. that was painstaking for even one year I will want various details about each player for 10-20 years.

\section{Scatter Plot}

\subsection{Scatter Plot}

The scatter plot visualization was used to demonstrate the average height and weight of the different offensive positions in the 2007 NFL Draft.  I am interested in this visualization because I want to analyze differences in football players.  As part of the my research, I want to analyze the difference between the player positions, understand the norms or these positions, and also any trends of how they are changing over time.   

\begin{figure}[h]
\centering
\includegraphics[width=0.5\linewidth]{scatter_plot.png}
\caption{\label{scatter_plot.png}Scatter Plot: Height vs Weight Offensive Players in 2007 NFL Draft.}
\end{figure}

\subsection{Histogram}

The Histogram is a visualization of the number of players drafted in the 2007 draft from each of the 7 rounds.  The data I used was an incomplete list. So, I was curious about this data.  Also, I wanted to learn more about what I could do with the data set and how that could be incorporated in a histogram. For future Research, I might make a variable that separates players by arm length and create a variable. I could set that as the x-axis and draft-round as the y axis.  That would be a very helpful visualization.  However, I did not have that specific data.  I used this part of the assignment to learn what I could do with a histogram.     

\begin{figure}[h]
\centering
\includegraphics[width=0.5\linewidth]{histogram.png}
\caption{\label{histogram.png}Histogram: Players Drafted Per Round:  2007 NFL Draft}
\end{figure}


\subsection{Box Plot}

The Box Plot demonstrates the Weight Range of each position in the 2007 NFL Draft. I found this interesting because it is yet another way I can visualize anthropometric measurements taken from data.  This is more for me to visualize the norms of each position.  These norms can be weight, height, body fat percentage, etc. If I have access to that data.     

\begin{figure}[h]
\centering
\includegraphics[width=0.5\linewidth]{boxplot.png}
\caption{\label{boxplot.png}Box Plot: Average weight Range Per Position in the 2007 NFL Draft}
\end{figure}
\end{document}